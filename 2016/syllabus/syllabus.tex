\documentclass[11pt]{amsart}
\usepackage{geometry}                % See geometry.pdf to learn the layout options. There are lots.
\geometry{letterpaper}                   % ... or a4paper or a5paper or ... 
%\geometry{landscape}                % Activate for for rotated page geometry
%\usepackage[parfill]{parskip}    % Activate to begin paragraphs with an empty line rather than an indent
\usepackage{graphicx}
\usepackage{amssymb}
\usepackage{epstopdf}
\usepackage{hyperref}
\usepackage{url}
%\DeclareGraphicsRule{.tif}{png}{.png}{`convert #1 `dirname #1`/`basename #1 .tif`.png}

\title{MUMT301 --- Music and the Internet}
\author{Gabriel Vigliensoni \\ September 8th, 2016}
%\date{February 29th, 2012}                                           % Activate to display a given date or no date

\begin{document}
\maketitle
%\section{}
%\subsection{}
\section{Course Description} 

The course will cover technologies and resources of the Internet (access tools, data formats, and media) and Web authoring (HTML/Javascript) for musicians; locating, retrieving, and working with information; putting information online; tools for music productivity, music research, music skills development, technology-enhanced learning, and promotion of music and musicians. 

The student will be evaluated on the quality of coding assignments, mid-term exam, oral presentation, class participation, and a final project.

%SPECIAL REQUIREMENTS: if your presentation requires specific equipment not found in the normal classroom, we can make arrangements---let us know in your project proposal.
%\subsection{Source material and contribution} 


\section{Instructor Contact}
\begin{description}
\item[Email]\href{mailto:gabriel@music.mcgill.ca}{gabriel@music.mcgill.ca}
\item [Phone]514.398.4535 ext. 0300
\item [Office]Room 508, 550 Sherbrooke West (Music Techonology Area)
\item [Office Hours]By appointment
\end{description}

\section{Course Detail}
\begin{description}
\item [Time]Thursday 4:35pm-7:25pm
\item [Place]E-215, Schulich School of Music (555 Sherbrooke West)
\item [Prerequisite]MUMT201 or MUMT202
\end{description}



\section{McGill Policies}

McGill University values academic integrity. Therefore all students must understand the meaning and consequences of cheating, plagiarism and other academic offences under the Code of Student Conduct and Disciplinary Procedures (see \href{http://www.mcgill.ca/students/srr/honest}{www.mcgill.ca/students/srr/honest} for more information).

In accord with McGill University’s Charter of Students’ Rights, students in this course have the right to submit in English or in French any written work that is to be graded. This right applies to all written work that is to be graded, from one-word answers to dissertations.


\section{Music Tech Computer Lab Policies and Access}
All registered students will automatically have access to room E-125--the Music Technology Computer Lab (MTCL).
The maximum storage capacity for your MTCL network account is 10 GB.\@
Unfortunately, there will be no backup for this account. If desired, use an external or flash drive. If you need to print something use Uprint from Library. For additional information about the MTCL rules and regulations see the lab's website: \href{https://mt.music.mcgill.ca/mtcl_rules_mtcl.htm}{(https://mt.music.mcgill.ca/mtcl\_rules\_mtcl.htm)}. The MTCL is available for all students registered in MUMT301 until the end of the term. Check the room's availability at the Booking System \href{(https://rbs.music.mcgill.ca/building/week.php?area=1&room=16)}{(https://rbs.music.mcgill.ca/building/week.php?area=1&room=16)}.

All registered students will receive access to the music.mcgill.ca server, as well as a music.mcgill.ca email account. Access to the server, as well as to the email account, will be valid until 30 August, 2017. After that date, the account will be automatically deactivated unless the student ask the LAN manager for a renewal of the account.

If you have any problem regarding your music.mcgill.ca account contact Alain Terriault, LAN Manager of the Schulich School of Music (\href{mailto:alain.terriault@mcgill.ca}{alain.terriault@mcgill.ca}). 

If you there are any MTCL computer-related issues you should contact Darryl Cameron, the MTCL technical manager (\href{mailto:darryl.cameron@mcgill.ca}{darryl.cameron@mcgill.ca}). 

For access-related issues ot the MTCL you must write an email to the Building Director's Office (\href{mailto:building.director@music.mcgill.ca}{building.director@music.mcgill.ca}) with a copy to the professor. 






\section{Course outline for Fall 2016}
\begin{itemize}
\item September 8th: Course introduction. New Music Economy. Introduction to UNIX/Linux commands. 
\item September 15th: History of the Internet. Introduction to HTML.\@
\item September 22nd: Internet technologies (e.g., Ethernet, TCP/IP, IP Addresses, DNS, DHCP\ldots) Introduction to CSS.\@
\item September 29th: File formats and compression (i.e., uncompressed, compressed, lossy, lossless) Introduction to Javascript.
\item October 6th: Music industry. Music distribution models. Javascript.
\item October 13th: \textbf{On-demand music streaming services presentation}. Javascript.
\item October 20th: Internet Radio. Streaming media softwares. The Infinite Dial 2015: Online radio and music discovery analysis. Intellectual property, copyright, and copyright alternatives. Mid-term preparation. Javascript.
\item October 27th: \textbf{Mid-term examination}. JavaScript.
\item November 3rd: Human and machine-driven music discovery tools. Music recommendation. JavaScript.
\item November 10th: APIs and web services. Query and results formats. Music APIs. Javascript and jQuery.
\item November 17th: Web-based sound generation. Web-based recording and sequencing applications. Web-based collaborative applications. Free sounds. Javascript.

% Music services for musicians, music companies, and advertisers. Metrics services. 

\item November 24th: Music scores and music libraries. Online music editors. Music Information Retrieval.\@ jqPlot and Musicmetrics API.\@

\item December 1th: \textbf{Final project proposal presentation. In-class help session for final project}
\item December 15th: \textbf{Final project submission}.
\end{itemize}

\section{Mark Distribution}
\begin{itemize}
\item 35\% Seven assignments (variable percentage each. See \href{https://mycourses2.mcgill.ca/d2l/lms/grades/my_grades/main.d2l?ou=230934}{Assignments tab} on MyCourses for details)
\item 15\% Mid-term exam
% \item 10\% Final exam
\item 40\% Final project (30\% project, 10\% presentation)
\item 10\% Participation
\end{itemize}

\section{Assignment Policy}
\begin{itemize}
\item All assignments are due as indicated on MyCourses (usually on Wednesday before the class)
\item Late assignments within 48 hours past the deadline will be given either D or F
\item Assignments submitted after 48 hours past the deadline will be given F
\end{itemize}

\section{Bibliography}
\begin{itemize}
\item Sterne, Jonathan. \emph{MP3: The meaning of a format.} Durham: Duke University Press, 2012.
\item Wikstr\"{o}m, Patrik. \emph{The Music Industry: music in the cloud.} Cambridge: Verlag Polity Press, 2009.
\item Links and electronic resources available in the \href{https://mycourses2.mcgill.ca/d2l/lms/links/view_links.d2l?ou=186686&d2l_stateScopes=%7B1%3A%5B%27gridpagenum%27,%27search%27,%27pagenum%27%5D,2%3A%5B%27lcs%27%5D,3%3A%5B%27grid%27,%27pagesize%27,%27htmleditor%27,%27hpg%27%5D%7D&d2l_stateGroups=%5B%27page%27%5D&d2l_statePageId=499&d2l_state_page=%7B%27Name%27%3A%27page%27,%27Controls%27%3A%5B%7B%27ControlId%27%3A%7B%27ID%27%3A%27slctlst_category%27%7D,%27StateType%27%3A%27%27,%27Key%27%3A%27%27,%27Name%27%3A%27category%27,%27State%27%3A%7B%27SelectedKey%27%3A%270%27,%27SelectedVal%27%3A%270%27%7D%7D%5D%7D&d2l_change=0#20825}{course webpage on MyCourses}

\end{itemize}

\section{Special Events}
\begin{itemize}

\item September 22nd: CIRMMT Distinguished Lecture: Tim Crawford, Computational Musicology, Goldsmiths University of London: \href{http://www.cirmmt.org/activities/distinguished-lectures/Tim_Crawford}{Busy Going Nowhere, or Learning to Live with Error? Personal reflections on three decades using computers with music.}
\item October 20th: CIRMMT Distinguished Lecture: Henkjan Honing, Music Cognition Group, University of Amsterdam, Netherlands: \href{http://www.cirmmt.org/activities/distinguished-lectures/copy_of_vorlander}{What makes us musical animals.}
\item November 17th: CIRMMT Distinguished Lecture: Frank Bedrossian

\end{itemize}


\end{document}  


